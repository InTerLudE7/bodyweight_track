\section{Monthly Summary}
\label{sec:Summary}
“Athlete Doppelgänger of the Month” is a light-hearted bit based solely on scale weight. It’s not a physique comparison—pros generally carry far less body fat.
\subsection{July 11 -- Aug 10}
\noindent{\color{red} Athlete Doppelganger of the Month: Sumo rikishi, Offensive Lineman.}

From July 11 to August 10, a total weight loss of 32~lbs was achieved, with body fat percentage decreasing from an estimated 33\% to a measured 29.6\%. Visceral fat rating dropped from 17 to 15, subcutaneous fat from 27.2\% to 25.3\%, and body water percentage increased from 49\% to 50.9\%. BMI declined from 37.2 to 32.6. Strength training volume was gradually increased, while aerobic intensity was maintained in the range of $10 < \mathrm{METs} < 12$. Skeletal muscle mass percentage showed a slight increase; however, when considered alongside the substantial overall weight reduction, some muscle loss was unavoidable. Other notable measurements revealed low bone density and a metabolic age rating exceeding chronological age. The low bone density is likely attributable to long-term caffeine intake and excessive consumption of carbonated beverages; future countermeasures will include calcium supplementation and weight-bearing impact activities such as running and jumping. The metabolic age issue is likely strongly correlated with current body fat percentage and will be reassessed after further reductions. Supplementation currently includes biotin with a multivitamin, high-purity curcuminoid turmeric, small amounts of ashwagandha and CoQ10, fish oil with meals, and calcium citrate with meals. Overall, cheat meals are acceptable for social occasions, but in terms of mitigating or preventing metabolic adaptation and satisfying psychological cravings for carbohydrates, a frequency of once per week is currently unnecessary.

\noindent{\color{red} Athlete Doppelganger of the Month: Offensive Lineman, Hammer Throw, Shot Put.}

\subsection{Aug 11 -- Sep 10}
\noindent{\color{red} Athlete Doppelganger of the Month: Offensive Lineman, Hammer Throw, Shot Put.}

From August 11 to September 10, atotal weight loss of 22 lbs was achieved, with body fat percentage decreasing from a measured 29.6\% to 24.4\%.  Visceral fat decreased from 16 to 12. During this period, I made several key adjustments to my daily schedule. I began tracking body measurements to complement weight tracking and decided to cancel the "Performance Tracking Day" since it did not fit well into my routine. I also experimented with two "Super Day," but the recovery burden was excessive—lasting up to 72 hours, so I will not continue with this approach. To improve nutrition, I added raw vegetables to ensure sufficient daily fiber intake and also limited caffeine consumption to no more than one coffee pod per day. Entering the fat-loss stage, water and mineral balance has stabilized, so body weight is no longer heavily affected by this short-term fluctuations. To support electrolyte balance and prevent hypokalemia, I also added one banana or apple to my daily diet. The target of fat-losing has been adjusted from Body weight 175 lbs to 175 lbs with 15\% Body fat.

\noindent The target body-fat rate, $t$, is set at $15\%$. Let $W$ denote the current body weight (lb), and $p$ the current body-fat percentage. With a fat-to-lean mass loss ratio $\rho$ : $(1-\rho)$, the following relationships hold:
\begin{equation*}
\begin{aligned}
\text{weight to lose (lb)} & =
W \times
\frac{p - 15}{100\rho - 15},\\
\text{fat to lose (lb)} & = \rho \times \text{weight to lose (lb)},\\
\text{LBM to lose (lb)} & = (1-\rho) \times \text{weight to lose (lb)}.
\end{aligned}
\end{equation*}

\begin{equation*}
\text{Validity condition: } \rho \geq 0.15.
\end{equation*}
Ideally, I want to keep $\rho \approx0.8$, but let's see how it works.

\noindent{\color{red} Athlete Doppelganger of the Month: 
Linebacker, Weightlifter, Judo, Wrestling.}

\subsection{Sep 11 -- Oct 10}
\noindent{\color{red} Athlete Doppelganger of the Month: 
Linebacker, Weightlifter, Judo, Wrestling.}
The previous goal setting was not very reasonable. To put it plainly, I had been treating 175 lbs at 15\% body fat as my target. That target would only be achievable under the assumption that weight loss consists of roughly an 80:20 ratio of fat to lean mass. However, there is a major misconception here: simply seeing both body weight and body-fat percentage decrease is already a very positive sign. Because of some external constraints, I need to keep the overall weight-loss period within 100 days; under that timeline, it becomes very hard to also insist on 15\% body fat. Therefore, this month I relaxed the 15\% requirement to a simpler rule---as long as body-fat percentage continues to trend downward, I am comfortable with whatever the final value is. At this point, I treat lowering body-fat percentage and reaching 175 lbs as my two sole criteria. 

More Meals added, calories and nutrition are split to two meals per day.

