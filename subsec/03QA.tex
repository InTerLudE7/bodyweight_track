\section{Hesitation and confirmation}
\label{sec:QA}
\subsection*{Personal Disclaimer}

\begin{quote}
This repository is my personal log of weight loss thoughts, choices, and experiences. It is \textbf{not} medical advice and I am not your clinician. Everyone’s physiology, history, and risks are different. What worked (or didn’t) for me may be unsafe or inappropriate for you.

Some entries mention symptoms, supplements, calorie targets, or training approaches that reflect \emph{my} context (age, sex, health history, labs, medications, schedule, and risk tolerance). Do not copy them blindly. If you have concerning symptoms—especially chest pain, severe shortness of breath, fainting/near-fainting, or new/worsening palpitations—seek medical care immediately.

All notes reflect my understanding at the time of writing and may change as I learn more; errors are possible. When in doubt, consult a qualified healthcare professional before making changes to diet, exercise, or medication.

Use this content at your own risk. No warranties or guarantees are provided.
\end{quote}

\subsection{Loose Skin:}
Regarding loose skin after weight loss: this issue is influenced by multiple factors, including age, lifestyle, diet, and genetics. My current priority is to lose weight as quickly as possible, with the goal of primarily reducing fat while minimizing muscle loss during the process. To help prevent loose skin, I am taking collagen peptides and vitamin C. At the same time, I am mentally prepared for the possibility of loose skin after completing my weight loss. If it occurs, I plan to give my skin about 18 months to recover naturally; if it does not, I will consider surgical treatment.

\subsection{Heart Palpitations:}
Regarding the issue of heart palpitations: at first, I was concerned that it might be caused by excessive exercise and long-time calorie deficit. However, after careful thinking, I concluded that it was most likely due to drinking too much coffee and a lack of sleep. To prevent palpitations from happening again, I plan to limit myself to at most one coffee pod per day and ensure at least seven hours of sleep. If maintaining seven hours of sleep becomes difficult, I will reduce the intensity of both strength training and cardio.

\subsection{Fatigue:}
Regarding the feeling of fatigue: in the past, I used to feel energized throughout the day. But now, I notice that once I study or work continuously for more than four hours, I become extremely exhausted and have to rest immediately. I actually see this as a positive change. For example, before I started losing weight, I found it hard to distinguish between craving and genuine hunger. After dieting, however, I can clearly tell the difference. Similarly, in the past, I couldn’t really distinguish between fatigue and boredom. Sometimes after working for a long time, I would feel “tired,” but in reality I was just bored—and scrolling through short videos only made me feel worse. Now I can clearly recognize when I am truly tired and need rest, and I can relieve it either by switching tasks or by taking a walk.

At the same time, I have always believed—though never scientifically verified—that fatigue is cumulative. I think the exhaustion from exercise, dieting, and studying all adds up together. Given my current situation, the most practical thing I can do is to minimize the fatigue caused by exercise.

\subsection{Exercise Exhaustion:}
Regarding managing exercise-induced fatigue, I believe the most important thing is to avoid “training to failure.” I also prioritize adjusting the order of my workouts. For me, doing weight lifting first and then cardio feels extremely exhausting, but starting with cardio and then moving on to strength work is much more manageable.

Another major contributor to fatigue is muscle soreness in individual muscle groups. To address this, I focus on compound free-weight exercises such as squats, deadlifts, farmer’s walk, and bench presses, while making sure not to train to failure. However, since my back is relatively weak, soreness there is still unavoidable.

In addition, injuries or discomfort caused by exercise also consume a lot of energy. For example, when I deadlift, calluses form on my hands, which is draining; to deal with this, I now use lifting straps and gloves. Similarly, I wear gloves to reduce pain during push-ups. As for knee pain from running, I switch to incline walking and extend the duration instead.

\subsection{Orthostatic Hypotension(OH):}
Around day 50 of fat loss, I noticed that the symptoms of OH became more pronounced. These episodes were not triggered by prolonged sitting, squatting, or lying down, but occurred during strength training—for example, when lowering the barbell or standing up after a plank. From a physiological perspective, this is more likely due to reduced blood volume rather than true hyponatremia. During an extended calorie deficit, glycogen stores become depleted, and since glycogen binds water, both total body water and circulating blood volume decrease. Combined with a relatively low-salt diet and increased water intake, this leads to lower blood pressure and a higher likelihood of OH.

Importantly, this should be considered a normal physiological response during the fat-loss phase rather than a pathological condition. A practical adjustment is to ensure adequate sodium and fluid balance, especially in the morning when the main meal is delayed until late afternoon (around 4–5 p.m.). Including a small sodium-containing snack earlier in the day can help stabilize blood pressure and reduce OH symptoms. My solution is to take BCAA mixed with collagen peptides before training.

\subsection{Hyponatremia and Hypokalemia:}
During the entire weight-loss phase, I only experienced hyponatremia once. This occurred after drinking 2 liters of iced coffee the night before and then exercising the following day. Essentially, because coffee acts as a diuretic, excessive intake led to a disturbance in water and electrolyte balance. However, through this incident, I gained a deeper understanding of sodium–potassium balance, and realized that hypokalemia—an even more dangerous condition that can be life-threatening—requires greater attention. As a solution, I started supplementing my diet with one apple or banana each day.

\subsection{Garbage Exercises:}
The term “garbage exercises” is often mentioned in fitness videos. It usually refers to movements that either fail to directly stimulate the target muscle group or carry a high risk of injury. Examples include: bent-over barbell rows (potential stress on the lower back), bench dips (potential shoulder strain), Arnold presses (instability that limits load), decline bench press (considered highly risky), paused deadlifts (disrupting the continuity of neural recruitment), as well as dumbbell kickbacks and dumbbell flyes (insufficient stimulation of the target muscles).

First, it’s important to clarify the purpose of resistance training. During a weight-loss phase, the priority is simply moving safely—any movement is better than none, as long as injury is avoided. For muscle building or bodybuilding, one should analyze the stimulation pattern based on kinesiology and biomechanics. For functional training or sports performance, the focus should shift toward sport-specific improvements.

Within these different contexts, I don’t deliberately emphasize avoiding so-called “garbage exercises.” My perspective is that one must first understand why an exercise is considered garbage, and then make appropriate modifications when using it. For instance, bent-over barbell rows can be replaced with bent-over dumbbell rows; paused deadlifts can be modified to traditional deadlifts with pauses during the eccentric phase; and bench dips can be adjusted with a smaller range of motion, focusing more on stability.

Finally, it’s crucial to note that for any exercise, precise understanding of the correct activation pattern and target muscle is essential. Otherwise, even a “good” exercise can turn into a “garbage” one.

\subsection{Weight Anchor Point:}
Regarding the concept of a “Weight Anchor Point,” this refers to the phenomenon where, after a successful weight loss, the body tends to return to its previous weight after a period of time—sometimes within a few months, sometimes over several years. From my own experience, between 2015 and 2025 I maintained a body weight above 260 lbs for about ten years, which formed a very pronounced Weight Anchor Point from a perceptual standpoint. I believe this ultimately comes down to whether my dietary and exercise habits align with my goals. As long as I avoid living according to the diet and activity patterns that sustained a weight above 260 lbs, this set point can be overcome. Of course, this also serves as an important reminder: even after successful weight loss, it is necessary to continue practicing lifestyle habits that are consistent with my target weight and body fat.

\subsection{Hormonal adaptations}
Hormonal adaptation is a normal physiological response. During prolonged caloric deficit, leptin typically decreases while ghrelin and cortisol tend to rise; with prolonged caloric surplus, estrogen may increase. When these shifts appear, I will adjust food intake by weighing both psychological stress and the degree to which training performance is being suppressed. 

Traditionally, strategies such as refeed weeks or diet breaks are used to mitigate these effects, but since I plan to keep the entire fat-loss phase within 100 days (July 11 – Oct 19), I do not intend to substantially raise calories for a refeed. After a marked downturn in condition on Sept 12–14, 2025—slow recovery, poor stamina, and intense hunger—I will add two eggs per day and a moderate portion of oats, and increase fruit to one apple and one banana per day.

\subsection{Chronotropic Incompetence}
“Chronotropic incompetence” basically means your heart can’t speed up the way it should when you exercise. That makes you get tired faster, feel weaker, and sometimes short of breath. If heart disease is ruled out, it’s usually because of an issue with the autonomic nervous system. Normally, when this happens, you should stop exercising right away, sit down or lie down, rest, and take some deep breaths. If you also have chest pain, real trouble breathing, dizziness, nausea, or feel like you might faint, you should get medical help immediately.

In my case, none of those red-flag is triggered. What I’m dealing with is more from a long period of running a 1000 calorie deficit. That kind of energy shortage pushes the body into “power-saving mode”: lowering metabolism, dialing down sympathetic activity, and giving more weight to parasympathetic tone. The result is a lower resting heart rate and less of a heart-rate rise during exercise, which makes workouts feel harder. Since it’s not really dangerous, the best approach is to scale back the workout intensity a bit and make sure I’m getting enough rest.